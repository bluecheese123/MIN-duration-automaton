\documentclass[a4paper,12pt]{article}
\usepackage[cm]{fullpage}
\usepackage[T1]{fontenc}
\usepackage{microtype, csquotes, lmodern}
\usepackage{amsmath,amsthm}
\usepackage{amssymb}
\usepackage{float}
\usepackage{xcolor}
\usepackage{mdframed}
\usepackage[shortlabels]{enumitem}
\usepackage{indentfirst}
\usepackage{hyperref,bm}
\usepackage{mathrsfs, mathtools}


\title{Minimal timed regular languages}
\author{Shree Ganesh}
\date{\today}

\newtheoremstyle{break}
  {\topsep}{\topsep}%
  {\itshape}{}%
  {\bfseries}{}%
  {\newline}{}%
\theoremstyle{break}

\newtheorem{theorem}{Theorem}[subsection]
\newtheorem{proposition}[theorem]{Proposition}
\newtheorem{lemma}[theorem]{Lemma}
\newtheorem{corollary}[theorem]{Corollary}
\newtheorem{claim}[theorem]{Claim}
\newtheorem{definition}[theorem]{Definition}
\newtheorem{example}[theorem]{Example}


\newcommand{\Z}{\mathbb{Z}}
\newcommand{\N}{\mathbb{N}}
\newcommand{\C}{\mathbb{C}}
\newcommand{\R}{\mathbb{R}}

\begin{document}

\maketitle
\titlepage
\newpage

\tableofcontents
\newpage

\section{Introduction}

Let $\Sigma$ be an alphabet. Let $\Sigma^*$ be the set of words over the alphabet $\Sigma$.
Partial orders on $\Sigma$ can be extended to a partial order on $\Sigma^*$ in several ways.
We'll describe a few natural ways to do that. 

\begin{definition}
    Let $(\Sigma,\leq)$ be a partially ordered finite alphabet. The \textbf{generalized prefix order}
    on $\Sigma^*$ is defined as follows: Let $u,v\in \Sigma^*$ with $u=a_1\dots a_k$, $v=b_1\dots b_l$.
    $u\leq v$ iff $k\leq l$ and $a_i\leq b_i\;\;\forall 1\leq i\leq k$.
\end{definition}


\begin{definition}
    Given a partial order '$\leq$' on $\Sigma^*$, for any $L\subseteq \Sigma^*$, we define $MIN(L):=
    \{w\in L|\;\; \text{there is no } x\in L\; \text{with } x\leq w\}$
\end{definition}

\subsection{Some commonly known orders}

We fix a partially ordered alphabet $(\Sigma,\geq)$.

\begin{definition}
    .
    \begin{enumerate}
        \item \textbf{Generalized dictionary order:} $u,v\in \Sigma^*$, $u\leq v$ iff $u$ is
        a prefix of $v$ or $u=pax$ and $b=pby$ where $p$ is the longest common prefix and $a,b\in\Sigma$, $a\leq b$.\\
        \item \textbf{Generalized lexical order:} $u,v\in \Sigma^*$, $u\leq v$ iff $|u|=|v|$ and $u\leq v$ with respect to the generalized dictionary order.
    \end{enumerate}
    
\end{definition}

\begin{theorem}
    Let $L$ be a regular language over $\Sigma$ which is recognized by a deterministic 
    finite automaton with $n$ states. Then,

    \begin{enumerate}
        \item With respect to the generalized dictionary order, $MIN(L)$ is recognized by a deterministic finite automaton with number of states $\leq (n-1)2^{n-2}+1$.
        \item With respect to the generalized "lexicographic" order, $MIN(L)$ is recognized by a deterministic finite automaton with number of states $O(n\cdot 2^n)$.      
    \end{enumerate}
\end{theorem}

The construction of these is as follows. Let $\mathcal{A}=(Q,\Sigma,\delta,q_0,F)$ be a deterministic finite automaton recognizing $L$. Let $<$ be an order on $\Sigma$.

\begin{enumerate}[(i)]
    \item When the order is extended to a generalized dictionary order, consider the following timed automaton. We can assume that all out-edges of the accept states of $\mathcal{A}$ have been dropped and that all accept states have been coalesced into a single accept state $f$. 
        {\begin{itemize}
            \item Set of states: $Q$.
            \item Transition function: 
                \[
                    \Delta(q,a)=
                    \begin{cases}
                        \delta(q,a) &\quad \text{if for all }b\in\Sigma,\delta(q,b)=\phi \text{ whenever }b<a\\
                        \phi &\quad \text{otherwise}. 
                    \end{cases}    
                \]
            \item Accept states: $\{f\}$.
        \end{itemize}}
    \item When we have a generalized "lexicographic" order instead, we have the following construction.
    {\begin{itemize}
        \item Set of states: $Q\times 2^Q$.
        \item Transition function: Given a state $(q,S)$ and a letter $a\in\Sigma$, define, $$S^{q,a}=\{\delta(q,b)\in Q:\; b\in\Sigma, b<a\}\cup\{\delta(p.c)\in Q:\; p\in S,c\in\Sigma,c\leq a\;\text{or }c>a\}$$
        Now, \[
                \Delta((q,S),a)=
                \begin{cases}
                    (\delta(q,a),S^{q,a}) &\quad \text{if } \delta(q,a)\in Q\\
                    \phi &\quad \text{otherwise} 
                \end{cases}    
            \]
        \item Accept states: $\{(q,S)|\; S\cap F=\phi\}$.
    \end{itemize}}
\end{enumerate} 

\section{The case of timed automata}

Here we explore ways to generalize above ideas to the case of timed automata. Extracting "minimal timed languages" with respect to the time can be particularly useful in synthesis problems.\\

Let $L$ be a timed language over the alphabet $\Sigma$. We introduce the following definitions.\\

Suppose $\prec$ is a partial order defined on $\Sigma$ and $<$ a partial order\footnote{We define it as a partial order instead of a total order to account for the setting we'll be working with } on the set of all time sequences $\{(\delta_1,\dots,\delta_n)|\;n\in \N, \delta_i\geq 0\}$.
\begin{definition}
    \begin{enumerate}
        \item Define $MIN(L)$ to be the language $\{(w,\tau)|\; \nexists (w',\tau')\in L \text{ such that }w'\prec w \}$.
        \item Define $MIN_T(L)$ to be the language $\{(w,\tau)|\; \nexists (w',\tau')\in L \text{ such that }\tau'\prec \tau\}$
    \end{enumerate}
\end{definition}

When analysing $MIN_T(L)$, we will primarily be working with the following orders on the time sequences. We denote the set of all time sequences by $T$ and for a given language $L$, the set of all time sequences $T=(\tau_1\dots,\tau_n)$ such that $(w,\tau)\in L$ by $T_L(w)$. We'll also use the orders $\prec$ and $<$ as defined previously as orders on $\Sigma$ and $T$ respectively.\\

\begin{theorem}
    Let $L$ be deterministic timed regular. Let the order $\prec$ be extended to the generalized dictionary (or subsequence, sebsegment, prefix, suffix) order on $\Sigma^*$. Then, 
    
    \begin{enumerate}
        \item $MIN(L)$ is determinisitc timed regular.
        \item $MIN_T(L)$ is deterministic timed regular when $<$ is the generalized "lexicographic" order on $T$.
    \end{enumerate}
    
\end{theorem}
\begin{proof}[\textbf{Proof of the first part:}]

    The proof of the second part requires a little bit more work and we'll build up to it after presenting a proof for the first part.
    Suppose there is a deterministic timed automaton $\mathcal{A}=(Q,\Sigma,\delta, X, q_0, F)$ recognizing $L$. Then, $MIN(L)$ is recognized by a deterministic timed automaton $\mathcal{A}'$ defined as follows. Since $UNTIME(L)$ is regular, $MIN(UNTIME(L))$ is also regular. Let $\mathcal{B}=(Q',\Sigma,\delta', q'_0, F')$ be a deterministic finite automaton recognizing $MIN(UNTIME(L))$. Let $\mathcal{A}'=(S,\Sigma,\Delta,s_0,X,\tilde{F})$ where,

    \begin{enumerate}
        \item $S=Q\times Q'$
        \item $((q,q'),a,(r,r'))\in\Delta$ with a set of clock constraints/reset $C$ iff $(q,a,r)\in\delta$ with a set of clock constraints/reset $C$ and $(q',a,r')\in\delta'$.
        \item $s_0=(1_0,q'_0)$.
        \item $\tilde{F}=F\times F'$.
    \end{enumerate}

    Then it's easy to see that $(w,\tau)$ is accepted by $\mathcal{A}'$ iff $(w,\tau)$ is accepted by $\mathcal{A}$ and $w$ is accepted by $\mathcal{B}$ iff $(w,\tau)\in MIN(L)$.
\end{proof}

Analysing the language $MIN_T(L)$ is slightly harder since here, we are restricting the language based on time duration constraints. Consider "$<$" to be the natural order on $R^{\geq 0}$. We can extend it to the lexicographic ordering on $T$. That is, given $\tau=(\tau_1,\dots,\tau_n)$ and $\delta=(\delta_1,\dots,\delta_n)$ for some $n\in \N$. Then, $\tau<\delta$ iff $\tau_i < \delta_i$ for all $1\leq i \leq n$.\\

\begin{lemma}
    Let $\mathcal{A}$ be a DTA accepting $L$ with max constant $M$ and integer constraints. Then, $(w,\tau)\in MIN_T(L) \implies$ $0\leq\tau_i\leq M$ and $\tau_i$ is an integer for all $1\leq i\leq n$.
\end{lemma}
\begin{proof}
    Firstly, if some $\tau_i> M$. Let the transition accepting $w_i$ in the run $\rho$ accepting $(w,\tau)$ be $(q,a,q')$ with some clock constraints $C$ and clock resets $[R]$. Let $r$ be the region in which the clock valuations $(x_1,x_2,\dots,x_n)$ is after the transition. Then, note that, setting $\tau_i=M$ also makes the clocks satisfy the constraints $C$ and after reset, end up in the same region $r$. Therefore, there is an accepting run of $(w,\tau_1\dots\tau_i'=M\tau_{i+1}\dots \tau_n')$ in $\mathcal{A}$ and $\tau'<\tau$ where $\tau'=\tau_1\dots\tau_i'=M\tau_{i+1}\dots \tau_n'$. This is a contradiction since $(w,\tau)$ is in $MIN_T(L)$. Therefore, for all $i$, $\tau_i\leq M$.\\

    Similar reasoning shows that $\tau_i$ is an integer for all $1\leq i\leq n$.
\end{proof}

Construct the following DTA $\mathcal{A}_I$ from $\mathcal{A}$. The only change in $\mathcal{A}_I$ that we make is the set of constraints. Suppose $x_i \in C_i$, $1\leq i\leq n$ be the set of constraints corresponding to a transition $t$. Then, we replace this by te set of constraints $x_i \in C_i\cap [2M]$\footnote{Here $[2M]$ is the set $\{1,2,\dots, M\}$}, $1\leq i\leq n$. This can be explicitly done by replacing $x_i\in C_i$ with $\wedge_{z\in C_i\cap [2M]} (x_i=z)$. Then, $MIN_T(L) = MIN_T(L(\mathcal{A}_I))$. This is because, from the previous lemma, every run in $\mathcal{A}$ accepting a minimal pair $(w,\tau)$ is such that each $\tau_i$ is integral and lesser than or equal to $M$. Therefore, the same run in $\mathcal{A}_I$ accepts $(w,\tau)$. On the other hand, $L(\mathcal{A}_I)\subseteq L$.\\

Now, we are set to prove the second part of our theorem. Note that the clock valuations and elapse times are all always integral during runs in $\mathcal{A}$ accepting minimal pairs $(w,\tau)$. So, we construct the following deterministic timed automaton from $\mathcal{A}$. Let $\Pi=\Sigma\times [M]$ and $\mathcal{B}$ be the timed automaton with the following description.
\begin{itemize}
    \item States: $Q\times [M]^{|X|}$.
    \item Alphabet: $\Pi$.
    \item Start state: $(q_0,\overline{0})$ (here, $\overline{0}$ is the zero vector representing the clock valuations).
    \item Accepting states: $F$.
    \item Set of clocks: $X$.
    \item Transitions: \{$((q,t_1),(a,\delta),(q',t_2), C, [R])$ for every transition $(q,a,q',C,[R])$ of $\mathcal{A}_I$ and every $t_1,t_2\in[M]^{|X|}$ and $\delta\in [M]$ such that $t_1+(\delta,\dots,\delta)=t_2$\}
\end{itemize}

Its easy to see that $UNTIME(\mathcal{B})=L(\mathcal{A}_I)$. And so, when $\Pi$ is given the partial order $\prec$ such that $(a,\delta_1)<(a,\delta_2)$ iff $\delta_1<\delta_2$ for all $a\in\Sigma$, this order can be generalize to the generalized dictionary order on $\Pi^*$ (which we will still denote by $\prec$ for convenience). Under this order, $MIN(UNTIME(\mathcal{B})=MIN_T(L(\mathcal{A}_I)))$. And therefore, $MIN_T(L(\mathcal{A}_I))=MIN_T(L(\mathcal{A}))$ is timed regular.
\end{document}